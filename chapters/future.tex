The improvements proposed in this thesis are just the first step towards
securing modern sytems.
Further effort is required to both enable developers to implement and 
adopt the proposed interfaces, 
and to keep these interfaces concurrent with ever-emerging threats.
Developers also need to adopt our proposed interfaces to their target
systems.
Future research can investigate how emerging hardware features can help 
improve our implementations of the proposed interfaces.
Finally, we describe other improvements to the interfaces investigated
in this thesis, and potential avenues to realize these improvements.

%%%%%%%%%%%%%%%%%%%%%%%%%%%%%%%%
\section{Beyond \midas: Double-fetch Protection and More}
\label{sec:conclusion:midas}
%%%%%%%%%%%%%%%%%%%%%%%%%%%%%%%%
While \midas provides systematic double-fetch protection to the 
kernel interface,
here we propose future use cases for \midas and 
suggest further improvements to its protection and performance 
guarantees.

Besides providing double-fetch protection, \midas can be used as a 
sanitizer for finding double-fetch bugs during dynamic testing of a
OS kernel and for enabling deep-argument checks for system call filters.
For example, \midas can be used as a sanitizer with 
Syzkaller~\cite{vyukov2019syzkaller} while fuzzing the kernel interface.
System call filters like SECCOMP add hooks to checks which validate
system call arguments before a system call executes, 
restricting processes from using potentially harmful system calls.
\midas can help fix a major shortcoming for modern system call
filtering mechanisms: the lack of deep-argument inspection.
Essentially, the filters cannot ``check'' any by-reference arguments stored
in user memory without creating a TOCTTOU instance since these arguments
will later be ``used'' by the kernel.
The introduced TOCTTOU bug makes the filtering check ineffective.
\midas, however, can be used to eliminate the TOCTTOU condition.
When running with \midas, the kernel would transparently create a snapshot 
when system call filters inspect userspace objects, with the same
snapshots serving data during the system call itself.
\midas, therefore, can make system call filtering more capable and
enable better isolation of userspace processes.

\midas will benefit from hardware protection for physical memory
ranges for preserving snapshots of accessed memory objects.
Particularly, \midas can reduce the cost of creating snapshots, 
which potentially incurs the cost of changing page table 
permissions across each virtual address space where an accessed page
is mapped, and the cost of corresponding TLB shootdowns.
Virtual page-based snapshotting also introduces false-positives due to
false-sharing of objects on the same page, as described in 
\autoref{sec:midas:design}.
New and upcoming hardware architectures can provide the physical
memory protection mechanisms required to implement snapshotting at low cost.
First, RISC-V's Physical Memory Protection (PMP) mechanism is ideal for
protecting a small number of physical memory ranges, only requiring updates
to per-core protection registers optionally accessible by the privileged
supervisor. 
PMPs are also flexible, allowing specifying protections for naturally aligned
power-of-2 address ranges enabling supervisors to protect memory at a
smaller granularity than pages.
However, PMP updates must be propagated across cores in order to protect
against cross-core attacks.
Midas can also benefit from architectures like Midgard, where a system wide
page table stores permissions to physical frames applicable across all
virtual address spaces.
Essentially, Midas can leverage Midgard's backside-page table entries to
make a single permission change for a snapshot.

Considering the increasing use of on- and off-chip accelerators accessible to
userspace applications, future work can extend \midas' protection to include
the threat of snapshot corruption through direct memory access (DMA)
from accelerators.
One strategy would be to use existing protection mechanisms for DMA, including
I/O Memory Management Units (IOMMUs) to protect snapshots.
Linux's reverse mapping must also be modified to include mappings for 
DMA-capable devices.

Hypervisor interfaces also require double-fetch protection, and can benefit 
from \midas' protection.
We leave the investigation of double-fetch protection of hypervisors to future
research.

%%%%%%%%%%%%%%%%%%%%%%%%%%%%%%%%
\section{Widespread Compartmentalization with \seccells}
\label{sec:conclusion:seccells}
%%%%%%%%%%%%%%%%%%%%%%%%%%%%%%%%
Following the eras of battling memory corruption, code injection and 
control-flow attacks,
we believe that the next frontier of defences lie with compartmentalization.
Particularly, this thesis highlights the need for isolation within
application components sharing a virtual memory space, and how the
architectural interface for virtual memory is crucial for supporting
isolation of memory accesses.

Ushering the era of well-compartmentalized applications requires 
secure policies for decomposing applications into compartments,
secure and performant mechanisms for implementing the policies, and
developer-friendly toolchains for supporting the process.
In this thesis, we propose \seccells as a supporting mechanism and
present a survey of related mechanisms in \autoref{ch:compreview}.
Future work in compartmentalization is required to address the
shortcomings of \seccells, implement \seccells as a back-end supporting
existing compartmentalization toolchains and
improve software support for \seccells.
We believe that parallel efforts into developing secure compartmentalization
policies for existing programs plays an equally crucial part in realizing
our vision of a compartmentalized future.

Our comparison of compartmentalization mechanisms shows potential
for improving \seccells.
Particularly, \seccells focuses on the issue of memory isolation.
Future work can add supervisor support for isolating kernel resources
to \seccells, perhaps involving system call filtering.
The granularity of passing data between compartments with zero-copy with
\seccells is a virtual memory area, which can be unsuitable for 
applications where compartments communicate with arguments held in 
fragmented data structures which do not map to a VMA.
Improvements can consider integrating CHERI's capability model for
fine-grained access to memory objects with \seccells to support cheap 
permission granting for complicated and fragmented data structures.
Finally, more research is required to investigate the requirement for
revocation of granted permissions, and the addition of this feature
to \seccells.
\seccells' cell description table, for example, can be used to encode
a VMA's owner, opening the potential for the implementation of an
unprivileged instruction allowing a VMA's owner to revoke another
compartment's permissions to that memory region.

Softare support for emerging architectures is crucial for their adoption.
We envision that future efforts will improve operating system support for
\seccells, along with porting of the requisite compiler toolchains and
libraries.
Automated or compiler-generated compartmentalization will be crucial to
compartmentalization efforts, and \seccells can be used as a backend to
tools like Enclosures~\cite{GhosnKPLB21}.