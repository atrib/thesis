"
It's a dangerous business, Frodo, going out your door. 
You step onto the road, and if you don't keep your feet, there's no knowing where you might be swept off to.
”, 
Bilbo Baggins often said to his young nephew Frodo Baggins. 
When I left India to start my Master's study at EPFL in 2016,
little did I know of the journey that lay ahead. 
Looking back at the path culminating in the attainment of the PhD, 
I must thank the many people who accompanied me through my wanderings, 
sweeping me off on unknown paths
and shining light  in times of darkness.

Every journey requires guidance, and I would like to first thank my academic advisors,
Prof. Mathias Payer and Prof. Babak Falsafi, who led me through my PhD.
Mathias has been a source of inspiration throughout the years,
with his belief that all systems are definitely and undeniably ``broken''.
Mathias was critical to my successes during the PhD, pushing me to try
every new avenue, teaching me to believe in my ideas, and providing the ever present
support required to realize ideas. 
My thesis is the culmination of a long series of ideas and ``weekend projects'' of 
increasing depth and clarity.
Many of the ideas comprising this thesis result from ideas refined and
polished with Mathias' invaluable insights.
Mathias' humanity was also key to surviving the unprecedented COVID-19 crisis
marking a major fraction of the PhD.
I am also deeply indebted to Babak, who hosted me in PARSA during my masters studies
in addition to being my PhD co-advisor. 
Babak is an exceptional leader, and has the ability to push research projects to look beyond the
superficial symptoms ailing computer systems, and find the right questions to ask.
Babak's belief and insistence on pushing for excellence has been instrumental in
deciding my research topics.

I would also like to thank the members of my
thesis committee, Prof. Carmela Troncoso, Prof. Sanidhya Kashyap, Dr. Anil Kurmus, 
and Dr. Anjo Vahldiek-Oberwagner.
I greatly appreciate their detailed and constructive feedback on my thesis draft,
and for providing a fresh perspective on my research allowing me to iron the final 
wrinkles in my research.
I would like to particularly thank Anjo for dedicating time to my thesis during a
time of personal difficulty.

I have also profited from a set of amazing collaborators helping me realize my
research ideas. 
Dr. Anil Kurmus' sabbatical at EPFL was an incredible opportunity to explore the
finer details of microarchitectural side-channel attacks.
I have also been fortunate to closely collaborate with other PhD students,
primarily Siddharth Gupta, Florian Hofhammer, Andres Sanchez, Yuanlong Li,
Uros Tesic and Lana Josipovic. 
Without our time spent working through tough problems, and working late to meet
ambitious deadlines, the PhD journey would have been particularly lacking lustre.
I would like to particularly thank Siddharth and Florian for teaching me to
structure papers into logical sequences of details from an unmanageable heap of
ideas and data.
I must also thank my student collaborators, Andrej Gorjan, Lucas Lopez Cendes, 
and Solène Husseini, who enabled me to 
explore the ideas which I never had time to do justice to. 
Alongside the students I had the pleasure to help during teaching assistant
duties, they helped me realize my love for teaching, for sharing knowledge,
and for learning together.

My love for teaching stems from the efforts of many amazing academics
who have injected me with their passion for knowledge.
I fondly remember the passionate high school lessons by 
Mr. Dey, Mr. Sur, Mr. Royan Mrs. Basu, and Mrs. Chakrabarthy.
They helped develop an adoration for maths, the sciences, and the arts.
My bachelors would have been far less educative without the efforts 
of Prof. Mainak Chowdhury, Prof. Aloke Dutta, Prof. Achla Raina,
and Prof. Suchitra Mathur.
The efforts of Prof. Chowdhury and Prof. Dutta formed the basis for my
future in low-level topics in computer science. 
Prof. Raina and Prof. Mathur fascinated me with the use of science to
understand society and particularly its language.
I am also thankful to my teachers at EPFL for the knowledge imparted during
numerous courses through my Masters and PhD studies here.
I have found the advice "prepare to throw one away" imparted during the
Principles of Computer Systems course invaluable for my PhD research.

I would like to specially extend thanks to Natascha Fontana for seamlessly 
managing the myriad administrative tasks and for dealing with EPFL's bureaucracy.
She managed to wrap her head around the interminable list of funding sources
supporting my PhD, and always managed to simplify requirements from our side.
Natascha's efforts made dealings with EPFL administration, from organizing and
cancelling visiting talks, reimbursing expenses, renewing contracts, 
hiring interns, and procuring hardware trivial.

No person walks through a PhD alone, and I am deeply grateful for my
many colleagues who walked beside me.
I would like to particularly thank the members of the Hive and PARSA, my two 
primary academic families during my time at EPFL. 
I am indebted to Florian Hofhammer, who has evolved from being a
student I supervised for a Master's thesis to a close collaborator to
eventually becoming a constant fixture in 
innumerable rambling discussions over coffee,
rants against every conceivable inconvenience, and 
being a rubber ducky for many random, ambitious or otherwise zany ideas.
Without your engineering skills, SecureCells may never have been a
paper, and I might not have finished a thesis.
Next, I have to thank Nicolas Badoux who has been an unending source of generosity, 
joy, cakes and chocolate across the years.
I must also thank Luca di Bartolomeo for teaching me to unabashedly ask "Why?",
and for teaching me to not feel guilty about being constructively competitive.
My gratitude also goes out to the other members of the Hive.
Ahmad Hazimeh has been my colleague in the lab for the longest, and has 
shared my earliest days here.
Andres Sanchez has been a close collaborator across numerous projects, and
will be remembered for his terrible puns.
The postdocs, Flavio Toffalini and Marcel Busch brought a fresh dose of life into the lab. 
I feel like Flavio always has a ear to listen to whatever I need to get off my mind.
I look forward to more afternoons playing badminton with the Hive's permanent
visitor, Qiang Liu.
Marcel has instrumental in creating many indelible memories, specially on Fridays.
I have also enjoyed spending time with the many other members of the lab, 
Antony Vennard, Lucio Romerio, Jelena Jankovic, Uros Tesic, Daniele Antonioli,
Chibin Zhang, Jean Michel, Han Zheng, Zhiyao Feng, and Phillip Mao.
On the PARSA side, I must thank Siddharth Gupta with
whom I shared many experiences including  
working on multiple projects and having a major skiing accident.
Sid also taught me how to write good papers.
Ognjen Glamocanin has been a constant presence throughout the years, and I 
look forward to sharing more burgers with you.
I must thank the remaining members, Alex Daglis, Mario Drumond, Arash Pourhabibi,
Mark Sutherland, Javier Picorel, Dimitrii Ustiugov,
Yuanlong Li, Shanqing Lin, Simla Harma, Yunho Oh, Rafael Pizzaro 
for welcoming me into the lab across many years.
I was also lucky to have an amazing cohort of other PhD peers.
Mahyar Emami was present at my first PC build, many intermediate Kashmiri lunches,
and will likely help me with my next PC build.
I will remember Sahand Kashani-Akhavan, the FPGA wizard, whose humour, 
humility and technical knowledge always left me astounded.
I must thank the many others with whom I spent innumerable light
moments in the fellowship room, Athanasios Xygkis, Xinrui Jia, 
Zeinab Shmeis, and Khashayar Barooti.
Finally, I must acknowledge the remaining systems PhDs who contributed to
making my PhD life amazing, Rishabh Iyer, David Aksun, Marios Kogias, Adrien Ghosn,
Neelu Kalani, Stefan Nikolic, Matthaios Olma, Stella Giannakopoulou,
Lana Josipovic,  and Andrea Guerrieri. 

Next, I want to thank the entire Zenith family, which was a crucial part of my
Master's and PhD life. 
The years in Marcelin were my best in Switzerland.
Particularly, I would like to thank Kirtan Padh, Kenneth Joseph Paul, 
Yassir el Maaroufi, Asli Yörüsün, Rahul Gupta, Rasool Ahmad, Sharbatanu Chatterjee, 
Shankha Nag, and Debdatta Ray.
The Zenith family was a crucial part of my life for many years, and helped support
me particularly through the COVID-19 pandemic.
The board game group comprising Kirtan Padh, Kenneth Joseph Paul, Christina Grimm,
Asli Yörüsün, Soner Serbest, and (more recently) Saskia Thomi have contributed to 
many periods of fun and intense competition. 
I would also like to thank Nina and Daniel Brissot, who welcomed me to Switzerland
and have always been supportive of my endeavors.

Finally, and most importantly, I must thank my family who have steadfastly been
the base on which each of my achievements is based.
This journey has been as much theirs as mine.
I must thank my parents, who taught me to pursue my interests and supported me through
thick and thin. 
My mother has inspired me to teach and to excel, teaching me how to find joy and 
happiness in every aspect of life. 
My father taught me to care, for everyone and anyone. 
I have rarely found people as compassionate as my father. 
I must also thank my aunt (Mashimoni) and uncle (Meshoi), 
who have cared for me like a son.
I cannot express my gratitude for my grandmother (Dida), grandfather (Dadabhai),
and grand-uncle (Chinida) who catalyzed my academic ambitions.
Finally, I must acknowledge my hero, my elder brother, has enabled each of my 
successes and has been a key ingredient to my efforts through more than 30 years.
There are no words to describe how he has helped me navigate each phase of my life. 
I doubt I would have progressed far beyond kindergarten without his active help.
I must thank the remaining two members of ``Chaarmurti'', my cousins.
They have always been a source of happiness and relaxation. 
Neel, Rishi, Dada, I look forward to more great times together. 

\newpage
I thank the various sources of funding supporting this thesis. 
My PhD research has been partially supported by 
EPFL, 
the 2020 IBM Research Fellowship, 
the European Research Council (grant agreement No. 850868),
DARPA (HR001119S0089-AMP-FP-034),  
ONR (award N00014- 18-1-2674 and 13000660-052),
the 2023 Qualcomm Innovation Fellowship,
Fondation Botnar, and 
a gift from Intel Corporation.

