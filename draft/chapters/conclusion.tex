Systems security is a continuous arms race between attackers and defences.
This thesis highlights two key interfaces in modern systems which lag
behind in this race, and proposes interfaces designed to mitigate attacks
exploiting the user-kernel system call interface, and the
userspace virtual memory interface.

\midas focuses on the vulnerability introduced by double-fetch bugs in
privileged software like OS kernels, and describes a systematic mitigation
mechanism to block this attack vector.
\midas identifies the implicit assumption underlying the existing system call 
interface's design and secures the interface by elevating the assumption to
an explicit guarantee.
This thesis also shows a practical implementation of \midas' design for
a popular OS running on commercial off-the-shelf hardware
In general, \midas highlights the security implications of implicit assumptions
made by system designers, and 
how changing computing systems can invalidate assumptions.
Comprehensive defence, instead, can be achieved through well-defined and
explicit security properties enforced across interfaces.
\midas guarantees a security invariant preserving the values of userspace data
objects accessed during system calls, ensuring that all reads to the same
objects return the same value.

\seccells investigates the mechanisms supporting userspace application
compartmentalization and makes the case for a mechanism enabling 
secure, performant and flexible intra-address space compartmentalization.
\seccells identifies the requirements supporting the three key application
objectives, and proposes a mechanism designed to support widespread
application compartmentalization.
\seccells provides strong isolation between compartments for data accesses
based on permissions stored in a permissions table storing
per-compartment per-memory region permissions.
\seccells introduces unprivileged instructions for implementing frequent 
operations in compartmentalized applications, like inter-compartment
control flow and zero-copy permission transfers at sub-microsecond time
scales while also implementing strict security conditions.
This thesis also describes our full-system prototype for \seccells based
on modified RISC-V RocketChip cores, the secure seL4 microkernel OS and
userspace benchmarks used for evaluating our design.
We are optimistic that \seccells will add momentum to the ongoing push
towards compartmentalization, improving interfaces within userspace applications
to reflect the varying trust relationships between application components.
This thesis also contributes a survey comparing state-of-the-art and
commercially used compartmentalization mechanisms.
This survey explores the design space across reviewed mechanisms and 
highlight the key security and performance ideals which mechanisms strive to
provide.

To support the ideal of open science, the code and other artifacts
supporting this thesis are available openly and freely.
Detailed documentation for \midas is maintained on the project's website
\url{https://hexhive.epfl.ch/midas}.
The evaluation results for \midas were submitted for artifact evaluation,
earning badges qualifying the artifact as ``Available'', ``Functional''
and ``Reproduced''.
\seccells{}' prototype, benchmarks, supporting infrastructure and
requisite documentation are also available at 
\url{https://www.hexhive.epfl.ch/securecells}.


