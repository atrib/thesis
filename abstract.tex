Computer systems rely heavily on abstraction to manage the exponential
growth of complexity across hardware and software.
Due to practical considerations of compatibility between components of
these complex systems across generations, developers have favoured stable 
interfaces at crucial boundaries such as between hardware and software, 
or between the kernel and userspace.
While these interfaces have persisted across more than 20 years, the modern 
computing
environment has evolved significantly in terms of security and performance.
Our increasingly connected systems share code components of widely varying
provenance and legacy interfaces are unable to counter modern threats
while maintaining strict performance objectives.
Computing requires new interfaces with stronger security guarantees which
can also support high performance applications.
First, the kernel-user interface remains one of the primary vectors for 
inter-application attacks
as compromising the kernel gives an attacker total control over the system's 
resources and to other applications on the same system.
Second, the virtual memory interface has newly emerged as another crucial
interface enabling attackers to remotely compromise systems
as applications increasingly execute third-party code, 
for example JavaScript scripts downloaded from the internet.
In this thesis, therefore, we investigate these two key interfaces to 
improve their security and performance limits.

The kernel-user system call interface suffers from double-fetch
bugs for passed-by-reference arguments stored in user memory.
Double fetches allow malicious users to compromise the isolation guaranteed 
at the kernel-user interface to illegally access memory,
cause kernel crashes, or to escalate their privileges.
The modern multi-user, multiprocessing environment allows the user to
change the arguments read by the kernel by modifying
the contents of memory from a concurrent thread.
Traditional testing techniques cannot eliminate all double-fetch bugs due 
to the complexity and configurability of the kernel.
The extensibility of the kernel further exacerbates the challenge as 
third-party modules loaded by the kernel may further introduce double-fetches.
We present \midas, a systematic mitigation for kernel double-fetches which
leverages the kernel's interface to read user memory to guarantee that 
every kernel read of a user object during a system call will return the 
same value.
\midas's guarantee makes an implicit assumption by kernel developers
explicit, protecting the kernel against a class of bugs while incurring
merely $3.4\%$ overhead on diverse workloads across the NPB and PTS 
benchmark suites.

Whereas modern systems software runs code from a plethora of sources with
varying degrees of trust, the traditional virtual memory abstraction lacks
support for isolating untrusted parts of an application within the same 
virtual address space.
Since all code running within a process execute at the same trust level,
buggy or malicious third-party code can compromise
the process by directly leaking or modifying memory used by other components 
of the application.
We must redesign the virtual memory interface to allow applications to
be compartmentalized, essentially implementing the principle of least privilege
by isolating untrusted parts of the application within compartments with 
limited access to the application's resources.
We present \seccells, a novel architectural interface for intra-address space
compartmentalization.
\seccells enables applications to define hardware-enforced memory views for 
application compartments with 
accelerated userspace instructions for inter-compartment calls.
In microbenchmarks, \seccells enables a 5-stage in-order core to switch 
compartments in only 8 cycles reducing the cost of transitions by an order
of magnitude compared to the state of the art.
We also build a full-system prototype of \seccells, based on the RISC-V 
RocketChip core running the seL4 kernel to evaluate userspace benchmarks.

This thesis also presents the first exhaustive comparison between 
compartmentalization mechanisms, evaluating both qualitative and quantitative
properties.
We describe relevant security and performance properties for practical
compartmentalization, and show how well each mechanism provides each
property.
A comprehensive review of compartmentalization techniques aims to enable
computer systems developers to define a secure, performant and usable interface
to support widespread compartmentalization of applications in the future.

This thesis posits that legacy interfaces between components of modern
computing systems inhibits their security guarantees, and explores issues
at two major interfaces.
We show that principled redesign of interfaces enables the implementation of
more secure systems while supporting high-performance application needs,
with the design and implementation of \midas and \seccells to tackle challenges
at the kernel-user and intra-process interfaces respectively.

